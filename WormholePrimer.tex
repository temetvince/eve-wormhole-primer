\documentclass[a4paper,12pt]{report}
\usepackage[parfill]{parskip}
\usepackage{geometry}
\usepackage{hyperref}
\usepackage{setspace}
\geometry{margin=1in}
\onehalfspacing

\title{Beyond the Event Horizon\\ \textit{\large A Primer for Wormhole Dwelling}}

\author{Lance Westar}
\date{\today}

\begin{document}

\maketitle

\tableofcontents
\newpage

\chapter{Introduction}

Navigating the vast and enigmatic expanse of wormhole space in EVE Online offers both significant challenges and rewards. This guide serves as a comprehensive resource for members embarking on wormhole operations, providing the essential knowledge to master this complex environment. Whether you're new to New Eden or an experienced pilot, this primer covers everything from fundamental preparation and daily routines to advanced strategies for survival, ISK generation, and corporate participation in wormhole space. After thoroughly reviewing this document, it is recommended that you seek mentorship from a seasoned corporation veteran to learn the art of wormhole rolling—a skill that, with practice, will distinguish you as a true follower of Bob.

\chapter{Getting Started}

\section{Preparations Before Departure}

Before venturing into wormhole space, it is crucial to undertake a few key preparatory actions to ensure your safety and efficiency.

Ensure your ship prominently displays the corporate symbol, signifying your membership and loyalty to the corporation. If you don't have the symbol yet, you can request it from any corporation member. It's usually best to place this symbol at the beginning of your ship's name, so consider saving it in the in-game Notebook for easy access.

Another vital step is to set your home station in a major market hub, such as Jita or Amarr, before entering wormhole space. Since respawning within wormhole space is impossible, having your home station in a well-connected hub provides a secure and convenient base of operations, ensuring you can quickly and safely return if needed.

\section{Actions Upon Arrival}

Once you’ve arrived in wormhole space, it is highly advisable to utilize a Jump Clone in the home hole to safeguard your primary implanted clone. Additionally, one of your first priorities should be to install Jump Clones in major market hubs. This setup allows you to swiftly return to your base if you are podded, minimizing downtime and enabling you to rejoin wormhole operations without delay.

\chapter{Daily Operational Routine}

\section{Logging In Procedures}

Upon logging in, it’s crucial to integrate smoothly with ongoing corporate operations. Start by immediately joining the corporate fleet, which is vital for coordinating activities in wormhole space. If the fleet doesn’t exist, take the initiative to create it. Next, log in to Pathfinder, a critical tool that provides valuable insights and data on wormhole space to support our operations. Finally, consider joining or starting a voice channel in the corporation’s Discord. Even if it’s initially empty, your presence can often encourage others to join.

\section{Securing the Home System}

Your daily routine in wormhole space begins with a few essential tasks that are crucial for your safety and the corporation’s success. Start by scanning for new signatures (sigs) in your home system. These signatures are key to wormhole exploration, and staying updated on any new developments is vital. If no new signatures are found in your home system, move on to the static wormholes and perform the same scan.

To streamline this process, use Pathfinder’s Signature Reader copy-and-paste feature. When consistently used across the corporation, this tool simplifies the tracking and management of PvE opportunities and enables quick and efficient signature checks, enhancing overall situational awareness.

\section{Expanding Territory}

Once your initial scans are complete, proceed with exploration. Choose an unscanned wormhole, ideally one that is a "leaf" in the network, suggesting it’s likely unexplored and could expand the corporation’s cartographic knowledge. Scan this wormhole to uncover hidden opportunities and potential threats.

\section{Maximizing Profit}

After exploration, consider engaging in PvE activities based on your scans. These might include gas harvesting, relic and data site exploration, or combat site engagement. Ensure you have the necessary skills and equipment before tackling each site. Avoid particularly challenging sites such as Ordinary, Bountiful, Vast, Vital, and Instrumental sites unless you are fully prepared.

\textsf{\textbf{TIP:}} \textsf{Respect your fellow corpmates by avoiding the practice of following another explorer and taking the sites they’ve uncovered, as this is considered poor etiquette. If you’re interested in a site, politely ask the prospector if they’d be willing to split the rewards for your assistance.}

\chapter{Earning ISK in Wormhole Space}

Earning ISK in wormhole space can be highly lucrative if you know where to focus your efforts. There are both consistent and inconsistent income sources, each offering varying levels of risk and reward.

\section{Consistent Income Sources}

If you’re seeking reliable ISK generation, focusing on consistent income sources is essential. The most dependable activities fall into two main categories: gas huffing and ratting. While mining and Planetary Interaction (PI) are also options, they tend to yield lower returns and are generally less lucrative compared to gas huffing and ratting.

\subsection{Gas Huffing}

Gas huffing involves extracting valuable gases from wormhole clouds and presents a balanced risk-to-reward ratio. To optimize your earnings, it's best to use T2 or Faction Prospect ships, or a Covetor equipped with mining boosts, particularly when operating in rich gas clouds. With the right setup, you can expect to earn approximately 100-150 million ISK per hour.

Be aware that NPC rats typically spawn 15-20 minutes after the initial warp to a gas site. Despite this, gas huffing remains a relatively safe and consistent method of generating income in wormhole space. After harvesting, the gas can be compressed for easier transportation—note that only the decompression process requires specific skills. Alternatively, the gas can be used in Reactions to create more profitable products, further increasing your ISK gains.

\subsection{Ratting}

Ratting, or engaging in combat with NPC pirates, is another consistent income source but comes with higher risks and potentially higher rewards.

\begin{itemize}
    \item \textbf{C1-C2 Wormholes}: These offer less profitability. If you have the option, focus on signature sites rather than anomalies, as the rewards for anomalies do not scale well with the lower wormhole classes.
    \item \textbf{C3 Wormholes}: These are particularly lucrative for ratting, with potential earnings of 200-250 million ISK per hour when using a Heavy Assault Cruiser. This income can nearly double if you pilot a Marauder, though this is generally not advised due to the increased risk.
    \item \textbf{C4 Wormholes}: Generally not recommended for ratting due to poor ISK-to-EHP ratios and difficult rat spawn ranges, making them less efficient and more challenging for sustained operations.
    \item \textbf{C5 Wormholes}: These are exceptionally profitable for those capable of soloing in Marauders or running a small fleet of three Leshaks or Nestors.
\end{itemize}

\section{Inconsistent Income Sources}

Inconsistent income sources in wormhole space offer potentially high rewards but come with varying degrees of risk. The primary activities in this category are relic and data site exploration.

\begin{itemize}
    \item \textbf{Relic Sites}: These sites generally offer higher rewards but are risky to explore due to the potential for hostile encounters.
    \item \textbf{Data Sites}: These offer lower rewards compared to relic sites but offer a variety of loot useful for industry.
\end{itemize}

\subsection{Core Node Placement Principles}

Understanding the placement of core nodes is key to successful site hacking. The core node, which you must reach to unlock the site, is always adjacent to at least one hostile node. Hostile nodes, which can damage your virus, never appear in positions surrounded by six other spots unless they are adjacent to the core. Typically, the core is located six or more spots away from your starting point, unless spatial constraints make this impossible. Being aware of these principles can help you anticipate the location of the core and plan your approach accordingly.

\subsection{Effective Hacking Strategies}

To maximize your hacking success, it’s important to use the right strategies:

\paragraph{Wrench Utility:} Utilize the wrench utility before your virus sustains damage to increase its health above the initial value. This can provide you with a significant advantage as you navigate through the nodes.

\paragraph{Shields and Kernel Rot:} Save the Shields and Kernel Rot items for critical moments. These powerful tools can make a crucial difference in challenging situations, giving you the upper hand when you need it most.

\paragraph{Path Selection:} If you encounter a particularly difficult hostile node, consider exploring alternative paths instead of forcing your way through. Sometimes, taking a different route can lead to a more favorable outcome.

\paragraph{Secondary Vectors Utility:} Deploy the "Secondary Vectors" utility to neutralize hostile nodes that reduce your attack power for three turns. This allows you to safely explore other nodes during this time, giving you more opportunities to find the core.

\subsection{General Hacking Tips}

\paragraph{Plan Ahead:} Before initiating a hack, take a few moments to assess the grid. Carefully survey the layout and identify potential paths, planning your approach to minimize risks and maximize efficiency.

\paragraph{Strategic Resource Management:} Your utilities are powerful tools, but they are limited. Use them strategically, reserving their use for critical moments when they can provide the most benefit. Effective resource management can often be the difference between success and failure.

\paragraph{Adaptability and Flexibility:} No two hacks are the same, and challenges can arise unexpectedly. Stay adaptable and be ready to change your strategy on the fly. Whether it's choosing an alternate path or deploying a utility at the right time, your ability to respond to changing circumstances is key to overcoming difficult hacks.

By following these principles and strategies, you can improve your success rate in relic and data site hacking and make the most of your exploration efforts in wormhole space.

\chapter{Bookmarking Best Practices}

Effective bookmarking is crucial for safe and efficient navigation in wormhole space. Properly managed bookmarks enable quick returns to key locations and help avoid potentially dangerous situations.

\section{Naming Conventions}

Establish and follow a consistent naming convention for your bookmarks to maintain clarity and organization. For example, use the format \texttt{.C4s SIG [Last 3 Digits of Name] EoL}, where \texttt{.} indicates a route leading homeward, \texttt{s} denotes a static wormhole, and \texttt{EoL} signals that the wormhole is nearing the end of its life. Consistent naming not only helps you quickly identify bookmarks but also aids in communication with other corporation members.

\section{Organizational Structure}

Organize your bookmarks into categories for easier access and management. Save exploration sites under a "Signatures" folder and wormholes under "Wormholes." This structure simplifies the process of locating specific bookmarks and ensures that your navigation data is logically arranged, reducing the chances of confusion during operations.

\section{Regular Maintenance}

Bookmarks should be regularly updated and pruned to maintain accuracy and relevance. As a general rule, avoid keeping bookmarks for more than two days unless necessary. You should set this when creating the bookmark for all signatures and wormholes. Regular maintenance of your bookmarks ensures that you always have current and reliable navigation information, minimizing the risk of following outdated paths.

\section{Operational Best Practices}

When traversing wormholes, always create a bookmark for the return path immediately. To improve accuracy, save wormhole bookmarks using the Overview or 3D view, as these methods reduce the likelihood of errors. For signatures, bookmarking from the Probe Scanner view is generally sufficient and efficient.

Before jumping into an unknown wormhole, right-click its bookmark, select Edit Location, and keep the editing window open during the jump. This allows you to quickly update the bookmark with any new information gained from the jump, ensuring that your navigation data remains precise and up-to-date.

\chapter{Understanding Wormholes}

Navigating wormhole space effectively requires a comprehensive understanding of wormhole mechanics, classifications, and the distinctive characteristics each wormhole possesses.

\section{Key Terminology}

Understanding key wormhole terminology is crucial for safe and effective navigation in wormhole space.

\paragraph{\texttt{K162}:} This designation is one of the most important in wormhole navigation. It indicates that a ship has entered the current wormhole from the opposite side, signaling that the wormhole has been activated. Identifying a \texttt{K162} wormhole is critical for assessing potential threats and gauging the level of activity in the surrounding area.

\paragraph{\texttt{Statics}:} Statics are wormholes that regenerate after collapsing, playing a significant role in wormhole dynamics. These wormholes will remain closed until they are warped to, so understanding their behavior is vital for planning navigation routes. If a static wormhole is only warped to and not traversed, it will stay closed for at least 9 hours, allowing for strategic planning and movement.

\paragraph{\texttt{Wandering}:} In addition to statics, wandering wormholes may also appear. These wormholes follow the same rules as statics when it comes to warping and traversing, adding another layer of complexity to wormhole navigation. Being aware of both static and wandering wormhole behaviors ensures better preparedness and decision-making in unpredictable environments.

\section{Security Classifications}

Wormholes are categorized by their security levels, which range from \texttt{Unknown} (C1-C3, C13) to \texttt{Dangerous Unknown} (C4-C5), and \texttt{Deadly Unknown} (C6). Wormholes leading to known space are classified by their destination: \texttt{Highsec}, \texttt{Lowsec}, and \texttt{Nullsec}, while wormholes leading to Triglavian space are classified as \texttt{Pochven}. Familiarity with these classifications is essential for gauging the risks and rewards associated with each wormhole.

\textsf{\textbf{TIP:}} \textsf{C1 and C3 wormholes don't have j-space statics and C4+ wormholes don't have k-space statics.}

\section{Visual Identification Tips}

Each wormhole class within the game is associated with a unique visual texture, which can be used for identification. To accurately determine a wormhole’s type, rotate the camera around the wormhole to carefully observe its distinct features. This methodical observation will enable you to quickly and accurately identify the class of wormhole you are dealing with, enhancing your situational awareness and decision-making in wormhole space.

\chapter{Survival Strategies}

Surviving in wormhole space demands unwavering vigilance and strategic foresight. Implementing the following strategies will significantly enhance your safety and minimize unnecessary risks.

\section{Operational Safety}

Maintaining operational safety is paramount in wormhole space. Always keep your ship aligned to a celestial object or a designated safe spot when you're not stationary. This practice ensures you can warp out instantly at the first sign of danger. Regularly use the directional scanner (D-Scan) to monitor for enemy ships and probes that might be nearby. Establish multiple safe spots within systems to observe your surroundings and retreat quickly if needed. Equipping cloaking devices on suitable ships further enhances your ability to avoid detection, while varying your travel routes reduces the risk of ambushes.

\section{Environmental Awareness}

Understanding the environment is crucial for making informed decisions. Nullsec wormhole connections generally experience lower traffic, making their holes more suitable for PvE activities. However, exercise caution when engaging in PvE within systems that have multiple \texttt{K162} connections, as these significantly increase the chances of encountering hostile players. Being aware of these environmental factors allows you to choose safer locations for your operations and reduces the likelihood of unexpected encounters.

\chapter{Market Hubs and Jump Clones}

Strategically placing your Jump Clones in major market hubs ensures that you can quickly return to key locations in known space. The most important market hubs in EVE Online are as follows:
\begin{enumerate}
    \item \textbf{Jita IV - Moon 4 - Caldari Navy Assembly Plant:} This is the premier trade hub in Caldari space and the most popular market in the game.
    \item \textbf{Amarr VIII (Oris) - Emperor Family Academy:} This is a major hub in Amarr space and serves as a secondary trading center.
    \item \textbf{Dodixie IX - Moon 20 - Federation Navy Assembly Plant:} This is the principal Gallente market hub.
    \item \textbf{Rens VI - Moon 8 - Brutor Tribe Treasury:} This is a key hub in Minmatar space.
    \item \textbf{Hek VIII - Moon 12 - Boundless Creation Factory:} This is another notable hub in Minmatar space.
\end{enumerate}

\chapter{Skill Training and Implants}

Optimizing your performance in wormhole space requires focused skill training and the strategic use of implants.

\section{Essential Skill Training}

Among the most crucial skills to develop is \texttt{Biology V}. This skill extends the duration of combat boosters and cerebral accelerators, offering substantial advantages in both combat and exploration scenarios. Investing in \texttt{Biology V} not only boosts your combat effectiveness but also prolongs the benefits of temporary enhancements, making it a key skill for sustained operations in wormhole space.

\section{Attribute-Enhancing Implants}

To further enhance your capabilities, consider equipping attribute-boosting implants that improve your overall performance:

\begin{itemize}
    \item \textbf{Ocular Filter} – Enhances perception.
    \item \textbf{Memory Augmentation} – Boosts memory.
    \item \textbf{Neural Boost} – Increases willpower.
    \item \textbf{Cybernetic Subprocessor} – Improves intelligence.
    \item \textbf{Social Adaptation Chip} – Augments charisma.
\end{itemize}

These implants provide a well-rounded boost to your core attributes, accelerating skill training and improving your effectiveness across various activities.

\section{Exploration-Focused Implants}

For those focused on exploration, the \texttt{Poteque ‘Prospector’} series of implants is invaluable. These implants include:

\begin{itemize}
    \item \textbf{Astrometric Rangefinding} – Enhances scan strength.
    \item \textbf{Astrometric Pinpointing} – Reduces scan deviation.
    \item \textbf{Astrometric Acquisition} – Decreases scan time.
\end{itemize}

Equipping these implants significantly enhances your scanning capabilities, allowing you to locate sites with greater speed and precision. For explorers, these implants are indispensable tools that improve both efficiency and success rates in wormhole space.

\chapter{Corporate Roles}

Each member of the corporation plays a specific role that contributes to the overall success of the group. Understanding and excelling in your role is key to thriving in wormhole space. The corporation has several key roles that members may fulfill:
\begin{itemize}
    \item \textbf{Scout:} As a scout, you are responsible for scanning and locating wormholes and sites. Your role is critical in mapping out the wormhole network and identifying potential threats and opportunities.
    \item \textbf{Combat Pilot:} Combat pilots engage in both PvE and PvP, defending the corporation's assets and initiating attacks on enemy forces.
    \item \textbf{Industrialist:} Industrialists focus on mining, manufacturing, and resource management, ensuring that the corporation has the materials and supplies needed for sustained operations.
    \item \textbf{Logistics Pilot:} Logistics pilots provide remote repairs and support during fleet operations, playing a vital role in keeping combat pilots and other members of the fleet alive.
    \item \textbf{Tackler:} Tacklers specialize in immobilizing enemy ships during engagements, preventing them from escaping and allowing the fleet to secure kills.
    \item \textbf{Bubbler:} Bubblers deploy warp disruption fields to prevent enemy ships from escaping, especially during large fleet engagements.
    \item \textbf{Electronic Warfare (EWar) Specialist:} EWar specialists disrupt enemy ship functionalities through various jamming techniques, weakening their combat effectiveness and giving the corporation a tactical advantage.
\end{itemize}

\chapter{Electronic Warfare (EWAR)}

Electronic Warfare (EWAR) is a critical aspect of fleet combat, allowing you to disrupt and neutralize enemy ships effectively. Each faction in EVE Online has a specific type of EWar that is most effective against it. The ship background color in the overview can help you identify the faction:
\begin{itemize}
    \item \textbf{Amarr (Yellow Background):} Use Radar jammers to disrupt Amarr ships.
    \item \textbf{Caldari (Blue Background):} Use Gravimetric jammers against Caldari ships.
    \item \textbf{Gallente (Turquoise Background):} Use Magnetometric jammers versus Gallente ships.
    \item \textbf{Minmatar (Red Background):} Use Ladar jammers to disrupt Minmatar ships.
\end{itemize}

\chapter{Defense and Tanking}

Optimizing your ship for combat requires a solid understanding of the preferred tanking methods and resistance vulnerabilities associated with each faction.

\section{Preferred Tanking Methods by Faction}

Each faction has a preferred tanking strategy that should guide your ship fittings:

\begin{itemize}
    \item \textbf{Amarr:} Armor tanking is the primary method for Amarr ships, focusing on strong armor resistances and repair capabilities.
    \item \textbf{Caldari:} Caldari ships excel with shield tanking, prioritizing high shield resistances and rapid shield regeneration.
    \item \textbf{Gallente:} Gallente ships often utilize structure tanking, relying on their robust hull strength to absorb damage.
    \item \textbf{Minmatar:} Speed tanking is the cornerstone of Minmatar defense, leveraging agility and high-speed maneuvers to avoid taking damage altogether.
\end{itemize}

\section{Resistance Weaknesses by Tank Type}

Each tank type has specific vulnerabilities that must be considered when outfitting your ship:

\begin{itemize}
    \item \textbf{Shield Tank:} Particularly susceptible to EM and Thermal damage, making it essential to bolster these resistances where possible.
    \item \textbf{Armor Tank:} Prone to Explosive and Kinetic damage, requiring extra attention to these resistances in your fittings.
\end{itemize}

\section{Exploiting Factional Resistance Weaknesses}

Understanding the resistance weaknesses of enemy factions allows you to tailor your offensive strategy for maximum effectiveness:

\begin{table}[h!]
\centering
\begin{tabular}{|c|c|c|}
\hline
\textbf{Faction} & \textbf{Primary Weakness} & \textbf{Secondary Weakness} \\ \hline
\textbf{Amarr} & Thermal damage & EM damage \\ \hline
\textbf{Caldari} & EM damage & Explosive damage \\ \hline
\textbf{Gallente} & Explosive damage & EM damage \\ \hline
\textbf{Minmatar} & Kinetic damage & Explosive damage \\ \hline
\end{tabular}
\end{table}

\chapter{Conclusion}

Wormhole space presents some of the most challenging and rewarding opportunities in EVE Online. Navigating its complexities requires not only skill and preparation but also a deep understanding of the mechanics, strategies, and roles that define successful operations in this unpredictable environment. 

By following the guidelines outlined in this primer, you will be well-equipped to thrive in wormhole space, whether you're engaging in exploration, combat, or signature activities. Remember that success in wormhole space is as much about teamwork and adaptability as it is about individual skill. Always remain vigilant, communicate effectively with your corporation, and continuously refine your strategies based on experience and learning.

As you venture beyond the event horizon, embrace the challenges that come your way, and let them sharpen your abilities and strengthen your resolve. The path to mastery in wormhole space is demanding, but with perseverance and the right knowledge, you will find great rewards in the depths of New Eden.

\end{document}
